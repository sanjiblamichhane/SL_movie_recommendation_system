\section{Need For Our Product}
According to Google Trend, the term ‘Ground Penetrating Radar” has been searched 2223 times in the last 12-months online. Among the related queries, ‘ground penetrating radar services near me’ and ‘ground-penetrating radar rental’ are the most common terms people searched on the internet. From this information alone, we can be assertive that there’s a need for our proposed product - an omnidirectional ground-penetrating radar robot(GPR).Therefore, we make a hypothesis that ground-penetrating radar robot is mostly demanded for infrastructure development; and, since there’s a significant number of online queries about rental GPR, we assume that the current market supplies an non-affordable product to purchase.

\subsection{Scholarly Journals and Similar Products}
%An article was published in IEEE Xplore, which features that 3D-underground mapping can be performed with a mobile robot and a GPR antenna \cite{kouros20183d}. The mobile robot towed the GPR antenna, which is mounted on a specifically designed trainer, is utilized as the means to cover the surface area while at the same time the antenna scanned the subsurface by emitting electromagnetic pulses. The gathered data are processed for the construction of a subsurface 3D map.\\ 
%This idea is a close resemblance to the product previously developed at City College \cite{feng2021gpr}. While Feng’s product was triangular in design, we intend to develop a four-wheel motor robot, which will be firm and smooth. With "Robotic Subsurface Mapping Using Ground Penetrating Radar" a paper from 1997, had already proved to us the potential of this in the market over time \cite{herman1997robotic}. "Over the last several decades, humans have buried a large amount of hazardous waste, unexploded ordnance, landmines, and other dangerous substances. During war periods, armies of different nations have buried millions of landmines around the world. A significant number of these landmines are still buried and active" \cite{lever2016robotic}. These buried objects are hard to relocate because accurate maps rarely exist, and in cases of landmines, the lack of a precise map is deliberate. Even if maps exist, the land may have significantly transformed the soil surface, making maps unusable. Often a solution for this will have a sensor that could sense these objects and have experts analyze the data to map underground position of the objects;however, this interpretation process could easily last for several days or weeks, depending on the size of the data and depending on how carefully the data are gathered .Additionally, an inaccurate registration will result in significant errors in the locations of the objects \cite{herman1997robotic}.\\
Hence, the manual scanning of subsurface is costly and filled with high risks. But with "automation could both lower the cost and shorten the time needed to perform the necessary steps."
